%---------- Inleiding ---------------------------------------------------------

\section{Introductie} % The \section*{} command stops section numbering
\label{sec:introductie}

%Hier introduceer je werk. Je hoeft hier nog niet te technisch te gaan.
%
%Je beschrijft zeker:

%\begin{itemize}
%  \item de probleemstelling en context
%  \item de motivatie en relevantie voor het onderzoek
%  \item de doelstelling en onderzoeksvraag/-vragen
%\end{itemize}

Agile is een framework dat in het laatste decennium aan een grote opmars bezig is. Meer en meer bedrijven stappen af van een inflexibel watervalsysteem ten voordele van het flexibele en dynamische Agile framework. Momenteel behoort het tot de standaarden van de industrie. Binnen software development elimineert het enkele van de meest courante problemen die verantwoordelijk zijn voor het falen van veel software projecten.

Het valt te bemerken dat er een grote variatie bestaat in hoe de principes van Agile exact worden toegepast. Frameworks zoals Scrum en Kanban omvatten dezelfde principes met een andere uitwerking. In dit werk zal er geprobeerd worden te achterhalen wat de voor- en nadelen van 2 dergelijke frameworks zijn en hoe deze gecombineerd kunnen worden teneinde een beter projectverloop.

%---------- Stand van zaken ---------------------------------------------------

\section{Stand van zaken}
\label{sec:state-of-the-art}

%Hier beschrijf je de \emph{state-of-the-art} rondom je gekozen onderzoeksdomein. Dit kan bijvoorbeeld een literatuurstudie zijn. Je mag de titel van deze sectie ook aanpassen (literatuurstudie, stand van zaken, enz.). Zijn er al gelijkaardige onderzoeken gevoerd? Wat concluderen ze? Wat is het verschil met jouw onderzoek? Wat is de relevantie met jouw onderzoek?
%
%Verwijs bij elke introductie van een term of bewering over het domein naar de vakliteratuur, bijvoorbeeld~\autocite{Doll1954}! Denk zeker goed na welke werken je refereert en waarom.
%
%% Voor literatuurverwijzingen zijn er twee belangrijke commando's:
%% \autocite{KEY} => (Auteur, jaartal) Gebruik dit als de naam van de auteur
%%   geen onderdeel is van de zin.
%% \textcite{KEY} => Auteur (jaartal)  Gebruik dit als de auteursnaam wel een
%%   functie heeft in de zin (bv. ``Uit onderzoek door Doll & Hill (1954) bleek
%%   ...'')
%
%Je mag gerust gebruik maken van subsecties in dit onderdeel.

De principes waar Agile op rust zijn relatief simpel. Deze werden in 2001 besproken en uitschreven door een groep van 17 developers \autocite{Beck2001}. De voor dit werk relevante principes zijn (geparafraseerd en vertaald):
\begin{itemize}
  \item Omarm veranderende requirements
  \item Lever regelmatig werkende software
  \item Een constant tempo moet voor onbepaalde tijd vol te houden zijn
  \item De beste architectuur, requirements en designs komen van zelf-organizerende teams
\end{itemize}



\emph{''Welcome changing requirements, even late in
development. Agile processes harness change for
the customer's competitive advantage''}. \autocite{Beck2001}. De klant is in staat om de processen van projecten te beheersen  aan de hand van on-site interactie en requirements geven de huidige noden van de eindgebruikers waarheidsgetrouw weer \autocite{Kumar2012}. Dit wordt gezien als het grootste voordeel van Agile ten opzichte van een klassiek watervalsysteem. Frequente communicatie met de opdrachtgever ('De klant') leidt tot een constante instroom van feedback. In combinatie met een ander voordeel ('Lever regelmatig werkende software', zie verder) is de analist beter in staat de wensen van de klant (of herzieningen hiervan) en de interpretatie van het ontwikkelingsteam ('De developers') gelijk te stellen (of hier zo dicht mogelijk bij te komen). Een bijkomend voordeel is hier dat er een constante herziening van het product is, wat zorgt dat foutopsporing sneller en gemakkelijk gebeurt \autocite{Imreh2011}.

\emph{''Deliver working software frequently, from a
couple of weeks to a couple of months, with a
preference to the shorter timescale''}\autocite{Beck2001}. Eén van de essenties van Agile werken, is het iteratief en incrementeel werken. In de praktijk vertaalt dit zich naar het werken in sprints.




%---------- Methodologie ------------------------------------------------------
\section{Methodologie}
\label{sec:methodologie}

Hier beschrijf je hoe je van plan bent het onderzoek te voeren. Welke onderzoekstechniek ga je toepassen om elk van je onderzoeksvragen te beantwoorden? Gebruik je hiervoor experimenten, vragenlijsten, simulaties? Je beschrijft ook al welke tools je denkt hiervoor te gebruiken of te ontwikkelen.

%---------- Verwachte resultaten ----------------------------------------------
\section{Verwachte resultaten}
\label{sec:verwachte_resultaten}

Hier beschrijf je welke resultaten je verwacht. Als je metingen en simulaties uitvoert, kan je hier al mock-ups maken van de grafieken samen met de verwachte conclusies. Benoem zeker al je assen en de stukken van de grafiek die je gaat gebruiken. Dit zorgt ervoor dat je concreet weet hoe je je data gaat moeten structureren.

%---------- Verwachte conclusies ----------------------------------------------
\section{Verwachte conclusies}
\label{sec:verwachte_conclusies}

Hier beschrijf je wat je verwacht uit je onderzoek, met de motivatie waarom. Het is \textbf{niet} erg indien uit je onderzoek andere resultaten en conclusies vloeien dan dat je hier beschrijft: het is dan juist interessant om te onderzoeken waarom jouw hypothesen niet overeenkomen met de resultaten.

